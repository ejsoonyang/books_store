\documentclass[a4paper]{article}
\usepackage[top=1in,bottom=1in,left=0.6in,right=0.6in]{geometry}

\usepackage{xeCJK}
\setCJKmainfont[RawFeature={vertical}]{TW-Kai}

\setCJKfamilyfont{kai-e}[RawFeature={vertical}]{TW-Kai-Ext-B}

\usepackage{setspace}
\setstretch{1.5}

\usepackage{graphicx}
\usepackage{wrapfig}

\pagestyle{empty}


\begin{document}
\begin{Huge}
㔾之字源

尹木彡\ 筆
\\

\end{Huge}
\begin{huge}
鑒於此問未能於網路得到一個很好的回答,本人在此試解一二,若有錯謬,請各學者朋友爲我糾正。
\\
\begin{wrapfigure}{l}{0cm}
\rotatebox{90}{\includegraphics{sli.png}}
\end{wrapfigure}
﹁卪卩﹂二者在篆書上已混用,查各種甲骨文和金文資料可得,卪爲信符,官員上朝時手握的竹片;卩爲跪人,古時多是跪坐,但卩之跪更有﹁跽﹂之內涵;亦有人認爲卩爲﹁厀﹂之初文,爲大腿關節處。

\rotatebox{90}{\includegraphics{khsu.png}}
\rotatebox{90}{\includegraphics{esu.png}}
\rotatebox{90}{\includegraphics{hesu.png}}
\rotatebox{90}{\includegraphics{hjju.png}}
\rotatebox{90}{\includegraphics{tesu.png}}

但㔾與卪卩並不等同,出自﹁犯氾笵範范﹂等字。可是用楷書查﹁㔾﹂,就會跳到卪的義項中。所以嚴謹的中研院已經把﹁犯﹂等字中的﹁㔾﹂用圖片來顯示。在楷書中混入卪的,我僅記得有﹁遷﹂。

本人未能查到㔾的獨立義項,觀與﹁犯﹂等字並存的篆書,到底﹁㔾﹂爲何義?爲什麼以這樣的形體展現?
\\

從﹁犯﹂在說文的釋義說起:︽說文︾:﹁犯,侵也。从犬,㔾聲。﹂由此,㔾有﹁侵犯虜掠﹂之義。

再看﹁氾﹂,濫也。這樣,㔾和監有著同樣的會意功能,意爲﹁關閉阻擋﹂。

﹁笵﹂之說文解爲﹁法︵灋︶﹂,那麼根據部件互釋的規律,氵與氵同,廌即竹刑,去與㔾互釋。這樣看來,㔾包含﹁行動走路﹂之義。

以上,側面談了㔾可能具有的三點義項。
\\

那麼組成篆體的這兩條畫綫到底爲何字何義?

\begin{wrapfigure}{l}{0cm}
\rotatebox{90}{\includegraphics{jbmm.png}}
\rotatebox{90}{\includegraphics{jbmc.png}}
\end{wrapfigure}
﹁直真﹂二字均含有乚。︽說文︾:﹁真,僊人變形而登天也。﹂說明這一條綫表示﹁昇天﹂,楷書寫作﹁乚﹂,爲﹁隱﹂之初文,而﹁隱﹂與﹁㥯︵去心︶﹂同。

\begin{wrapfigure}{l}{0cm}
\rotatebox{90}{\includegraphics{stkr.png}}
\end{wrapfigure}
再看﹁匿﹂字,︽說文︾:﹁匿,亡也。从匸,若聲,讀如羊騶箠。﹂這裡的匸,爲乚加一指事,指其消失的地方。

有人記得﹁千與千尋﹂的日文片名嗎?﹁千と 千尋の神隠し﹂,隠和し在一起,是否也是在說乚就是隱義呢?
\\

\begin{wrapfigure}{l}{0cm}
\rotatebox{90}{\includegraphics{lu2.png}}
\end{wrapfigure}
以上,我個人認定,㔾篆中的長綫爲乚,︽說文︾:“﹁乚,匿也。象︵
\CJKfamily{kai-e}
𨒅
\rmfamily
︶曲隱蔽形。讀若隱。﹂則乚有﹁昇天歸隱﹂義。

然則﹁㔾﹂除﹁乚﹂外還有另一條折綫,何解?
\\

在篆體中,倒寫反寫是常用到的構字方法,如﹁人亡﹂二字就多有倒寫爲﹁匕仄﹂。﹁眞﹂之上即爲反人。

\begin{wrapfigure}{l}{0cm}
\rotatebox{90}{\includegraphics{vhiil.png}}
\end{wrapfigure}
由﹁鄉﹂之篆體得,皀之左右爲邑,一正一反,乡其實是邑的鏡像。﹁鄉﹂就是兩人對坐吃飯。

\begin{wrapfigure}{l}{0cm}
\rotatebox{90}{\includegraphics{niq.png}}
\end{wrapfigure}
再看﹁舛﹂,其實就是兩夊,一正一反,夊爲有阻慢行義,兩個糾結的人,一慢一快,方向相反,這樣結果就錯綜復雜。如﹁粼韋﹂等字就有﹁凌亂﹂、﹁相違﹂之義。
\\

至此,我個人認爲,﹁㔾﹂其實就是兩個乚,一正一反。解爲某人正想歸隱昇天,但又受阻而返。如此,﹁犯笵氾﹂等字便能夠釋明。
\\

另:﹁㔾﹂是否正因爲其由兩個乚組成,本身並不成字,所以以篆體作字頭的︽說文︾未給出單獨的義項?
\\

又:既然乚是昇天,在書寫篆書的時候,是不是應該從下往上畫呢?
\\
\\
\\

create by \LaTeX , and 請問怎麼把xelatex也生成像\LaTeX 一樣的形狀?

\end{huge}
\end{document}
